\documentclass{article}
\usepackage{mathtools}
\usepackage{color}
\usepackage[utf8]{inputenc}
\usepackage{mathtools}
\usepackage{xcolor,colortbl}
\definecolor{newred}{HTML}{e74c3c}
\definecolor{newblue}{HTML}{3498db}
\definecolor{newpurple}{HTML}{9b59b6}
\newcommand{\wombat}{\cellcolor{newred}}
\newcommand{\pit}{\cellcolor{newblue}}
\newcommand{\wompit}{\cellcolor{newpurple}}

\begin{document}
\title{TDT4136 - Assignment 2}
\author{Filip F Egge}
\date{October 4, 2013}
\maketitle

\newpage
\section*{Task 1 - Models and Entailment in Propositional Logic}
\subsection*{1)}
	\begin{center}
		\begin{tabular}{|m{0.25cm}|m{0.25cm}|m{0.25cm}|}
			\hline
			- &&  \\ \hline
			S & - &\\ \hline
			- & B & \wombat \\ \hline
		\end{tabular}
		\quad
		\begin{tabular}{|m{0.25cm}|m{0.25cm}|m{0.25cm}|}
			\hline
			- &&  \\ \hline
			S & \wombat & \\ \hline
			- & B & - \\ \hline
		\end{tabular}
		\quad
		\begin{tabular}{|m{0.25cm}|m{0.25cm}|m{0.25cm}|}
			\hline
			\wombat &&  \\ \hline
			S & - & \\ \hline
			- & B & - \\ \hline
		\end{tabular}
		\quad
		\begin{tabular}{|m{0.25cm}|m{0.25cm}|m{0.25cm}|}
			\hline
			- &&  \\ \hline
			S & - & \\ \hline
			- & B & - \\ \hline
		\end{tabular}
	\end{center}

	\begin{center}
		\begin{tabular}{|m{0.25cm}|m{0.25cm}|m{0.25cm}|}
			\hline
			- &&  \\ \hline
			S & - &\\ \hline
			- & B & \wompit \\ \hline
		\end{tabular}
		\quad
		\begin{tabular}{|m{0.25cm}|m{0.25cm}|m{0.25cm}|}
			\hline
			- &&  \\ \hline
			S & \wombat & \\ \hline
			- & B & \pit \\ \hline
		\end{tabular}
		\quad
		\begin{tabular}{|m{0.25cm}|m{0.25cm}|m{0.25cm}|}
			\hline
			\wombat &&  \\ \hline
			S & - & \\ \hline
			- & B & \pit \\ \hline
		\end{tabular}
		\quad
		\begin{tabular}{|m{0.25cm}|m{0.25cm}|m{0.25cm}|}
			\hline
			- &&  \\ \hline
			S & - & \\ \hline
			- & B & \pit \\ \hline
		\end{tabular}
	\end{center}

	\begin{center}
		\begin{tabular}{|m{0.25cm}|m{0.25cm}|m{0.25cm}|}
			\hline
			\pit &&  \\ \hline
			S & - &\\ \hline
			- & B & \wombat \\ \hline
		\end{tabular}
		\quad
		\begin{tabular}{|m{0.25cm}|m{0.25cm}|m{0.25cm}|}
			\hline
			\pit &&  \\ \hline
			S & \wombat & \\ \hline
			- & B & - \\ \hline
		\end{tabular}
		\quad
		\begin{tabular}{|m{0.25cm}|m{0.25cm}|m{0.25cm}|}
			\hline
			\wompit &&  \\ \hline
			S & - & \\ \hline
			- & B & - \\ \hline
		\end{tabular}
		\quad
		\begin{tabular}{|m{0.25cm}|m{0.25cm}|m{0.25cm}|}
			\hline
			\pit &&  \\ \hline
			S & - & \\ \hline
			- & B & - \\ \hline
		\end{tabular}
	\end{center}

	\begin{center}
		\begin{tabular}{|m{0.25cm}|m{0.25cm}|m{0.25cm}|}
			\hline
			\pit &&  \\ \hline
			S & - &\\ \hline
			- & B & \wompit \\ \hline
		\end{tabular}
		\quad
		\begin{tabular}{|m{0.25cm}|m{0.25cm}|m{0.25cm}|}
			\hline
			\pit &&  \\ \hline
			S & \wombat & \\ \hline
			- & B & \pit \\ \hline
		\end{tabular}
		\quad
		\begin{tabular}{|m{0.25cm}|m{0.25cm}|m{0.25cm}|}
			\hline
			\wompit &&  \\ \hline
			S & - & \\ \hline
			- & B & \pit \\ \hline
		\end{tabular}
		\quad
		\begin{tabular}{|m{0.25cm}|m{0.25cm}|m{0.25cm}|}
			\hline
			\pit &&  \\ \hline
			S & - & \\ \hline
			- & B & \pit \\ \hline
		\end{tabular}
	\end{center}

	\begin{center}
		\begin{tabular}{|m{0.25cm}|m{0.25cm}|m{0.25cm}|}
			\hline
			- &&  \\ \hline
			S & \pit &\\ \hline
			- & B & \wombat \\ \hline
		\end{tabular}
		\quad
		\begin{tabular}{|m{0.25cm}|m{0.25cm}|m{0.25cm}|}
			\hline
			- &&  \\ \hline
			S & \wompit & \\ \hline
			- & B & - \\ \hline
		\end{tabular}
		\quad
		\begin{tabular}{|m{0.25cm}|m{0.25cm}|m{0.25cm}|}
			\hline
			\wombat &&  \\ \hline
			S & \pit & \\ \hline
			- & B & - \\ \hline
		\end{tabular}
		\quad
		\begin{tabular}{|m{0.25cm}|m{0.25cm}|m{0.25cm}|}
			\hline
			- &&  \\ \hline
			S & \pit & \\ \hline
			- & B & - \\ \hline
		\end{tabular}
	\end{center}

	\begin{center}
		\begin{tabular}{|m{0.25cm}|m{0.25cm}|m{0.25cm}|}
			\hline
			- &&  \\ \hline
			S & \pit &\\ \hline
			- & B & \wompit \\ \hline
		\end{tabular}
		\quad
		\begin{tabular}{|m{0.25cm}|m{0.25cm}|m{0.25cm}|}
			\hline
			- &&  \\ \hline
			S & \wompit & \\ \hline
			- & B & \pit \\ \hline
		\end{tabular}
		\quad
		\begin{tabular}{|m{0.25cm}|m{0.25cm}|m{0.25cm}|}
			\hline
			\wombat &&  \\ \hline
			S & \pit & \\ \hline
			- & B & \pit \\ \hline
		\end{tabular}
		\quad
		\begin{tabular}{|m{0.25cm}|m{0.25cm}|m{0.25cm}|}
			\hline
			- &&  \\ \hline
			S & \pit & \\ \hline
			- & B & \pit \\ \hline
		\end{tabular}
	\end{center}

	\begin{center}
		\begin{tabular}{|m{0.25cm}|m{0.25cm}|m{0.25cm}|}
			\hline
			\pit &&  \\ \hline
			S & \pit &\\ \hline
			- & B & \wombat \\ \hline
		\end{tabular}
		\quad
		\begin{tabular}{|m{0.25cm}|m{0.25cm}|m{0.25cm}|}
			\hline
			\pit &&  \\ \hline
			S & \wompit & \\ \hline
			- & B & - \\ \hline
		\end{tabular}
		\quad
		\begin{tabular}{|m{0.25cm}|m{0.25cm}|m{0.25cm}|}
			\hline
			\wompit &&  \\ \hline
			S & \pit & \\ \hline
			- & B & - \\ \hline
		\end{tabular}
		\quad
		\begin{tabular}{|m{0.25cm}|m{0.25cm}|m{0.25cm}|}
			\hline
			\pit &&  \\ \hline
			S & \pit & \\ \hline
			- & B & - \\ \hline
		\end{tabular}
	\end{center}

	\begin{center}
		\begin{tabular}{|m{0.25cm}|m{0.25cm}|m{0.25cm}|}
			\hline
			\pit &&  \\ \hline
			S & \pit &\\ \hline
			- & B & \wompit \\ \hline
		\end{tabular}
		\quad
		\begin{tabular}{|m{0.25cm}|m{0.25cm}|m{0.25cm}|}
			\hline
			\pit &&  \\ \hline
			S & \wompit & \\ \hline
			- & B & \pit \\ \hline
		\end{tabular}
		\quad
		\begin{tabular}{|m{0.25cm}|m{0.25cm}|m{0.25cm}|}
			\hline
			\wompit &&  \\ \hline
			S & \pit & \\ \hline
			- & B & \pit \\ \hline
		\end{tabular}
		\quad
		\begin{tabular}{|m{0.25cm}|m{0.25cm}|m{0.25cm}|}
			\hline
			\pit &&  \\ \hline
			S & \pit & \\ \hline
			- & B & \pit \\ \hline
		\end{tabular}
	\end{center}

	The tables above show all the possible worlds. The red square tells us that there is a wumpus occupying that square. The blue square means a pit is located on that square. And purple means that both a wampus and a pit is on that square. The four first rows matches  	\(\alpha_2 = \)``There is no pit in [2,2].'' and the third column matches \(\alpha_3 =\) ``There is a wumpus in [1,3]''. The \textbf{KB} is true for the table in row 2, column 3.

\subsection*{2) Exercise 7.4}
	\subsubsection*{a) \(False \models True\)}
	False entails True, this is correct
	\subsubsection*{b) \(True \models False\)}
	False
	\subsubsection*{c) \((A \wedge B) \models (A \Leftrightarrow B)\)}
	\subsubsection*{d) \((A \Leftrightarrow B) \models (A \lor B)\)}
	\subsubsection*{e) \((A \Leftrightarrow B) \models (\lnot A \Leftrightarrow B)\)}
	\subsubsection*{f) \((A \lor B) \wedge (\lnot C \lor \lnot D \lor E) \models (A \lor B \lor C) \lor (B \lor C \lor (D \to E))\)}
	\subsubsection*{g) \((A \lor B) \wedge (\lnot C \lor \lnot D \lor E) \models (A \lor B) \wedge (\lnot D \lor E)\)}
	\subsubsection*{h) \((A \lor B) \wedge \lnot(A \to B)\)}
	\subsubsection*{i) \((A \lor B) \to C \models \lnot(A \to B)\)}
	\subsubsection*{j) \((C \lor (\lnot A \wedge \lnot B)) \equiv ((A \to C) \wedge (B \to C))\)}
	\subsubsection*{k) \((A \Leftrightarrow B) \wedge (\lnot A \lor B)\)}
	\subsubsection*{l) \((A \Leftrightarrow B) \Leftrightarrow C \) has the same number of models as \((A \Leftrightarrow B)\) for any fixed set of proposition symbols that includes A, B, C.}

\subsection*{3) Exercise 7.7}
	\subsubsection*{a) \(B \lor C\)}
		There are 3 models in which this sentence is true
	\subsubsection*{b) \(\lnot A \lor  \lnot B \lor \lnot C \lor \lnot D\)}
		Using a truth table i have found the number of models to be 15.
	\subsubsection*{b) \((A \to B) \lor A \lor  \lnot B \lor C \lor D\)}
		This sentence is always false and has no models.

\subsection*{4)}
\subsection*{5)}
	\subsubsection*{a) \(A_1 \lor A_{73}\)}
		\(3/4 + 2^{100}\)
	\subsubsection*{b) \(A_7 \lor (A_{19} \wedge A_{33})\)}
		\(5/8 * 2^{100}\)
	\subsubsection*{c) \(A_{11} \to A_{22}\)}
		\(3/4 + 2^{100}\)
\section*{Task 2 - Resolution in Propositional Logic}
\subsection*{1) - Convert each of the following sentences to Cunjunctive Normal Form (CNF)}
	\subsubsection*{a) \(A \wedge B \wedge C\)}
	Already in CNF
	\subsubsection*{a) \(A \lor B \lor C\)}
	Already in CNF
	\subsubsection*{a) \(A \to (B \lor C)\)}
	\(\lnot A \lor (A \lor B)\)
\subsection*{2) - Consider the following Knowledge Base (KB):}
	\begin{itemize}
		\item \((A \lor \lnot B) \to \lnot C\)
		\item \((D \wedge E) \to C\)
		\item \(A \wedge D\)
	\end{itemize}
	Use resolution to show that \(KB \models \lnot E\)
	\\
	The first step is to convert the knowledge base into Cunjunctive Normal Form.
	\begin{itemize}
		\item \((\lnot A \lor \lnot C) \wedge (\lnot C \lor E)\)
		\item \(C \lor \lnot D \lor \lnot E\)
		\item \(A \wedge D\)
	\end{itemize}

\end{document}